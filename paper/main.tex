\documentclass{article}
\usepackage{arxiv}

\usepackage[utf8]{inputenc}
\usepackage[english, russian]{babel}
\usepackage[T1]{fontenc}
\usepackage{url}
\usepackage{booktabs}
\usepackage{amsfonts}
\usepackage{nicefrac}
\usepackage{microtype}
\usepackage{lipsum}
\usepackage{graphicx}
\usepackage{natbib}
\usepackage{doi}



\title{Детекция эмоций. Сравнение и анализ классических методов машинного обучения и методов обучения с трансформерами}

\author{ Панин Никита Александрович \\
        Факультет вычислительной математики и кибернетики \\
        МГУ им. Ломоносова \\
        \texttt{s02200456@gse.cs.msu.ru} \\
	%% examples of more authors
	\And
	д.ф-м.н., профессор, Воронцов Константин Вячеславович \\
        Факультет вычислительной математики и кибернетики \\
        МГУ им. Ломоносова \\
        \texttt{vokov@forecsys.ru} \\
	%% \AND
	%% Coauthor \\
	%% Affiliation \\
	%% Address \\
	%% \texttt{email} \\
	%% \And
	%% Coauthor \\
	%% Affiliation \\
	%% Address \\
	%% \texttt{email} \\
	%% \And
	%% Coauthor \\
	%% Affiliation \\
	%% Address \\
	%% \texttt{email} \\
}

\date{}

\renewcommand{\shorttitle}{\textit{arXiv} Template}

%%% Add PDF metadata to help others organize their library
%%% Once the PDF is generated, you can check the metadata with
%%% $ pdfinfo template.pdf
\hypersetup{
pdftitle={Детекция эмоций. Сравнение и анализ классических методов машинного обучения и методов обучения с трансформерами},
pdfsubject={q-bio.NC, q-bio.QM},
pdfauthor={Воронцов Константин Вячеславович,Панин Никита Александрович },
pdfkeywords={},
}

\begin{document}
\maketitle

\begin{abstract}
	В работе была рассмотрена задача детекции эмоций на датасете, в основу которого вошел WASSA датасет из твитов для детекции эмоций. На выходе алгоритма классификации эмоций в твитах была одна из 5 эмоций: нейтральная эмоция, грусть, страх, радость, гнев. Были применены различные методы "классического" машинного обучения, такие как, SVM, логистическая регрессия, метод k-ближайших соседей и наивный байесовский классификатор. Также классификация эмоций была проведена с помощью файн-тюнинга нескольких версий BERT. Основной целью работы являлось проведение сравнительного анализа для классических моделей машинного обучения(wKNN, Multinomial Bayes Classifier, Logistic Regression, SVM) и для моделей глубокого обучения (в качестве предобученной модели брались BERT, RoBERTa, BERTweet и их large-версии). В результате исследования было показано, что по метрике accuracy для моделей классического обучения c tf-idf векторизацией текстов лучше всего работает SVM с RBF ядром (accuracy ≈ 0.8387 на тесте), а наиболее качественные результаты получаются с помощью предложенной в исследовании модели с предобученным BERTweet (accuracy ≈0.88 на тесте).
\end{abstract}


\keywords{Детекция эмоций \and  NLP}

\section{Введение}
В нашем эмоционально насыщенном мире разумно стремиться к пониманию тонких чувств и настроений, выраженных в текстах. С развитием машинного обучения и искусственного интеллекта, возникает возможность автоматизировать и усовершенствовать процессы анализа и интерпретации текстов, выявляя эмоции, ассоциированные с ними. Детекция эмоций в текстах стала актуальным направлением исследований в области обработки естественного языка (NLP) и анализа тональности и привлекает все большее внимание в последние два десятилетия~\cite{affectdetectionintexts, jiawenfuji}. Термины "распознавание эмоций"({\itshape{emotion detection}}) и  "анализ тональности"({\itshape{sentiment analysis}}) часто используются как взаимозаменяемые, хотя между этими двумя понятиями существуют очевидные различия~\cite{yadollahi}. Анализ тональности в основном измеряет субъективное отношение с точки зрения полярности настроения: нейтральное, положительное, негативное. Выявление эмоций предполагает идентификацию более детальных эмоциональных состояний, например,  счастье, гнев, страх, удивление. 

Эмоции имеют множество применений в разных сферах.  В маркетинге анализ предпочтений потребителей помогает улучшить бизнес-стратегии~\cite{cambria}. В социальных сетях распознавание агрессивных эмоций  помогает выявить потенциальных преступников или террористов~\cite{cheong}. Мониторинг эмоций в реальном времени на основе данных социальных сетей может помочь в профилактике самоубийств~\cite{ren}. Определение эмоций во время кризисов или катастроф позволяет понять чувства людей по отношению к конкретной ситуации, что способствует управлению в кризис и принятию важных решений~\cite{ahmad}. 


\subsection{Постановка задачи}
В  данной работе была рассмотрена задача детекции эмоций на датасете, в основу которого вошел WASSA датасет из твитов для детекции эмоций~\cite{wassa}. На выходе алгоритма классификации эмоций в твитах была одна из 5 эмоций: нейтральная эмоция, грусть, страх, радость, гнев. Были применены различные методы "классического" \ машинного обучения, такие как, SVM, логистическая регрессия, метод k-ближайших
соседей и наивный байесовский классификатор. Также классификация эмоций была проведена с помощью файн-тюнинга нескольких версий BERT. За основу бралась статья~\cite{albu}. Основной целью работы являлось проведение сравнительного анализа полученных моделей. Также в работе предлагается сравнение новых моделей, таких как large-версии бертов, логистическая регрессия, wKNN, SVM с другими ядрами, с результатами из основной статьи.

\subsection{Существующие решения}


Ванг и др.~\cite{wang} создали большой набор данных твитов, используя хэштеги эмоций. Они применили два различных классификатора, логистическую регрессию и Naïve Bayes, чтобы исследовать эффективность различных признаков, таких как n-граммы, лексикон эмоций и информация о части речи, для задачи идентификации эмоций. Наибольшая достигнутая точность составила 0,6557.


Мохаммад~\cite{mohammad} создал корпус твитов, маркированных эмоциями, используя хэштеги. Он применил бинарные SVM, по одному для каждой из шести основных эмоций Экмана~\cite{ekman}, и использовал наличие или отсутствие униграмм и биграмм в качестве бинарных признаков. Бинарные классификаторы смогли предсказать эмоции со сбалансированным F1-score 0,499.


Янссенс и др.~\cite{janssens} исследовали влияние использования слабых меток по сравнению с сильными метками на распознавание эмоций для корпуса, состоящего из 341 931 твита. Слабые метки были созданы путем использования хэштегов твитов, а сильные метки - с помощью краудсорсинга. Характеристики, извлеченные путем объединения n-грамм и TF-IDF (Term Frequency-Inverse Document Frequency), были применены к пяти алгоритмам классификации: Стохастический градиентный спуск, SVM, Naïve Bayes, Nearest Centroid и Ridge. Результаты показали снижение F1-score на 9,25\% при использовании слабых меток.

К сожалению, классические методы машинного обучения не могут учесть последовательную природу текста, поэтому некоторые модели глубокого обучения, такие как рекуррентные нейронные сети (RNN), LSTM~\cite{hochreiter} и GRU~\cite{cho}, стали более перспективными в определении эмоций в тексте~\cite{kratzwald, chatterjee, xu}. Хотя рекуррентные модели принимают во внимание последовательный характер текста и показывают передовые результаты для различных задач NLP, они обладают некоторыми слабостями: медленная скорость, необходимость обучения с нуля и ограниченная способность улавливать долгосрочные зависимости в тексте~\cite{hochreiter}. Также требуется большой объем размеченных данных для обучения. Подготовка большого объема размеченных данных является трудоемкой и дорогостоящей процедурой, и именно здесь вступает в игру перенос обучения(transfer learning). С его помощью можно добиться лучших результатов по сравнению с традиционными моделями глубокого обучения с гораздо меньшим количеством обучающей выборки. Предварительно обученные языковые модели, такие как BERT~\cite{devlin} (Bidirectional Encoder Representations from Transformers) и его варианты, OpenAI GPT (Generative Pre-trained Transformer)~\cite{radford} и Transformer-XL~\cite{dai}, получили широкое распространение в различных задачах NLP и продемонстрировали впечатляющие результаты.

 Некоторые работы используют предварительно обученные языковые модели для классификации эмоций или тональности в тексте. Например, исследование~\cite{prottasha} использует BERT в качестве слоя эмбеддинга, после которого выводы передаются через слои CNN и BiLSTM для анализа настроений на бенгальском языке.




\bibliographystyle{plain}
\bibliography{references}

\end{document}
